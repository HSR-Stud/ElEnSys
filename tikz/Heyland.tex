
\begin{tikzpicture}

	% Bekante Punkte
	\coordinate (pI0) at (1.7,0.4); % I_0
	\coordinate (pPa) at (19:20); % P_A
	\coordinate (pVS) at (0,2);     % Verlust im Stator
	\coordinate (pPosVerlust) at (15,0); % Position an der der Verlust eingetragen wird
	\def\direction{55}; %Winkel von I_N
	
	%Errechnete Punkte
	\coordinate (dirpIn) at ({cos(\direction)},1);
	
	% Koordinatensystem
 	\draw[->] (-1,0) -- ++(23,0) node[below] {$-j$};
  	\draw[->] (0,-1) -- ++(0,17) node[right] {$re$};
  	\path[name path=abszisse] (0,0) -- +(23,0);
  	
  	\clip (-1,-1) rectangle (23,17.5);  % Ganzer Bereich, damit keine Linien weiter gehen.
  	
  	\draw[color=blue, ->] (0,0) -- +(0,10) node[left] {$U_{ph}$};
  
  	% PA und Leistungsline
  	\draw[->] (0,0) -- (pPa) node[above, left=5] {$I_K$};
  	\node at (pPa) [right=1.2cm, above = 1.2] {Anlaufpunkt};
  	\node at (pPa) [right=0.2cm] {$P_A \quad s=1$};
  	\draw[name path=leistungslinie] (pI0) -- (pPa) node[below, near start, sloped] {Leistungslinie};
  	
  	\draw[color=red, ->] (0,0) -- +(pI0) node[below=0.5cm, midway] {$I_0$};
  	\draw[dashed, name path=eisenverlustlinie] (pI0) -- +(20,0);
  	

  	
  	% Mittelsenkrechte auf der Leistungslinie
  	\begin{scope}
  		\coordinate (pMSLL) at ($ (pI0)!0.5!(pPa) $);
  		\coordinate (pMSLL2) at ($ (pMSLL)!10cm!90:(pI0) $);
  		\path[name path=mittelsenkrechte] (pMSLL) -- (pMSLL2);
  		\path [name intersections={of= mittelsenkrechte and eisenverlustlinie}] (intersection-1) coordinate (pM);
  		\draw[dashed, shorten >=-10pt] (pMSLL) -- (pM);
  		\node[draw, blue, name path global=kreis] at (pM) [circle through=(pI0)] {};
  		\path (pM) node[below] {$M$};
  		\draw ($(pMSLL)!3mm!(pMSLL2)$) to[bend right=45] ($(pMSLL)!3mm!(pPa)$);
  		\path ($(pMSLL)!1.4mm!(pMSLL2)!1.4mm!(pPa)$) node {.};
	\end{scope}
	
	% P_infty berechnen und zeichenen
	\begin{scope}
		\coordinate (pSchnitt) at ($ (pI0) + (pVS) + (pPosVerlust) $);
		\path[name path global= momentlinie] (pI0) -- ($(pI0)!2!(pSchnitt)$);
		\path[name intersections={of= kreis and momentlinie, name=s}] (s-1) coordinate (pPinf);
		\draw[->] (pI0) -- (pPinf) node[near start, below, sloped] {Momentenlinie};
		\draw[dashed] (pM) -- (pPinf);
		\path (pPinf) node[above = 0.4cm, right=0.2cm] {Bremspunkt};
		\path (pPinf) node[right=0.2cm] {$P_\infty \quad s=\infty$};		
	\end{scope}
	
	% Senkrechte auf MP_infty
	\begin{scope}
		\coordinate (pMMP) at ($ (pM)!0.67!(pPinf) $);
  		\coordinate (pMMP2) at ($ (pMMP)!16cm!90:(pPinf) $);
  		\draw[shorten < = -10cm, ->, name path=skala] (pMMP) -- (pMMP2);
  		\path[name intersections={of= skala and momentlinie, name=sk0}] (sk0-1) coordinate (pSkala0);
  		\draw ($(pMMP)!-3mm!(pMMP2)$) to[bend right=45] ($(pMMP)!3mm!(pPinf)$);
  		\path ($(pMMP)!-1.4mm!(pMMP2)!1.4mm!(pPinf)$) node {.};
	\end{scope}
	
	% Verlängerung von P_inf-Pa
	\begin{scope}		
		\draw[dashed, name path=PinfPa] (pPinf) -- ($(pPinf)!5!(pPa)$);
		\path[name intersections={of=PinfPa and skala, name=sk1}] (sk1-1) coordinate (pSkala1);	
	\end{scope}
	
	% Skala
	\begin{scope}
		\foreach \x in {0.0,0.1,0.2,0.3,0.4,0.5,0.6,0.7,0.8,0.9,1.0} {
			\draw ($(pSkala0)!\x!(pSkala1)$) node[left=2] {\x};
			\draw ($(pSkala0)!\x!(pSkala1)- (-.1,0)$) -- ($(pSkala0)!\x!(pSkala1)- (.1,0)$);
		}
	\end{scope}
	
	% M - S_k
	\begin{scope}
		\coordinate (pSSK) at ($(pI0)!(pM)!(pPinf)$);
		\path[name path=skline] (pM) -- ($(pM)!17cm!(pSSK)$);
		\path[name intersections={of= kreis and skline, name=intersSk}] (intersSk-1) coordinate (pSk);
		\draw[dashed] (pM) -- (pSk) node[above=0.5cm] {Kipppunkt};
		\draw[dashed] (pM) -- (pSk) node[above] {$S_K$};
		\draw ($(pSSK)!3mm!(pM)$) to[bend right=45] ($(pSSK)!3mm!(pPinf)$);
  		\path ($(pSSK)!1.4mm!(pM)!1.4mm!(pPinf)$) node {.};
	\end{scope}
	
	% Beschriftung P_zu0
	\coordinate (pPzu00) at ($ (pI0)+ (4,0) $);
	\coordinate (pPzu01) at ($ (0,0)!(pPzu00)!(15,0) $);
	\draw[<->] (pPzu00) -- (pPzu01) node[midway, right] {$P_{zu_0}$};
	
	% Beschriftung M_K
	\path[name path=mkbeschriftung] (pSk) -- +(0,-20);
	\draw[name intersections={of= mkbeschriftung and momentlinie, name=intersMk}, <->] (pSk) -- (intersMk-1) node[midway, left] {$M_K$};

	% I_N
	\begin{scope}
		\coordinate (pINN) at (dirpIn);%($(pI0)!(pM)!(pPinf)$);
		\path[name path=inline] (0,0) -- ($(0,0)!17cm!(pINN)$);
		\path[name intersections={of= kreis and inline, name=intersIn}] (intersIn-1) coordinate (pIN);
		\draw (0,0) -- (pIN) node[left] {$I_N$};
	\end{scope}
	\draw [->] (0,3) arc (90:{\direction+5}:3) node[below left, below = 0.5cm, left = 0.7cm] {$\varphi$};
	% Beschriftung Pmech
	\path[name path=pmech] (pIN) -- +(0,-20);
	\path[name intersections={of= pmech and leistungslinie, name=intersPmech}] (intersPmech-1) coordinate (pPmech);
	\draw [<->] (pIN) -- (pPmech) node[midway, right] {$P_{mech}$};
	%Beschriftung PV
	\path[name path=pv] (pPmech) -- +(0,-20);
	\path[name intersections={of= pv and abszisse, name=intersPv}] (intersPv-1) coordinate (pPv);
	\draw [<->] (pPmech) -- (pPv) node[above right] {$P_V$};
	
  	%Senkrechte von PA herunter
  	\path[name path=pVRotor] (pPa) -- +(0,-20);
	\path[name intersections={of= pVRotor and momentlinie, name=intersPVRotor}] (intersPVRotor-1) coordinate (pPVRotor);
	\draw [<->] (pPa) -- (pPVRotor) node[midway, right] {Stromwärmeverlust Rotor} node[midway, right=1.2cm, below=0.7] {Anlaufmoment} ;
  	%Senkrechte von PA herunter Teil 2
  	\path[name path=pVStator] (pPVRotor) -- +(0,-20);
	\path[name intersections={of= pVStator and eisenverlustlinie, name=intersPVStator}] (intersPVStator-1) coordinate (pPVStator);
	\draw [<->] (pPVRotor) -- (pPVStator) node[midway, right] {Stromwärmeverlust Stator};
	 %Senkrechte von PA herunter Teil 3
  	\path[name path=pVEisen] (pPVStator) -- +(0,-20);
	\path[name intersections={of= pVEisen and abszisse, name=intersPVEisen}] (intersPVEisen-1) coordinate (pPVEisen);
	\draw [<->] (pPVStator) -- (pPVEisen) node[midway, right] {Eisenverluste};
	
	
\end{tikzpicture}